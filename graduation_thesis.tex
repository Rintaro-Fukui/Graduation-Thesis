\documentclass[a4paper,12pt]{report}

% Package declarations
\usepackage{graphicx}
\usepackage{amsmath}
\usepackage{amssymb}
\usepackage{hyperref}
\usepackage{geometry}
\usepackage{fancyhdr}

\geometry{left=25mm,right=25mm,top=25mm,bottom=25mm}

% Customization for headers and footers
\pagestyle{fancy}
\fancyhead[L]{}
\fancyhead[C]{}
\fancyhead[R]{\thepage}
\fancyfoot[L]{}
\fancyfoot[C]{}
\fancyfoot[R]{}

\title{卒業論文\\ タイトルタイトルタイトル}
\author{武蔵野太郎\\ データサイエンス学科\\ 武蔵野大学}
\date{2025年1月}

\begin{document}

% Title Page
\maketitle

% Abstract
\chapter*{要旨}
「ダミーテキスト」近年,プロの作家やライターではない,...

% Table of Contents
\tableofcontents

% Chapter 1: Introduction
\chapter{序論}
このファイルは,武蔵野大学データサイエンス学部データサイエンス学科の卒論のテンプレートです....

\section{論文の構成}
論文の構成を以下に述べる.2章では...

% Chapter 2: Figures, Tables, and Equations
\chapter{図と表と数式}
\section{図表の配置}
図表は本文中に,出現する場所に配置します.

\section{図表のキャプション}
図と表にはキャプションを付けます.キャプションというのは...

\subsection{図とそのキャプションの例}
\begin{figure}[htbp]
    \centering
    \includegraphics[width=0.7\textwidth]{example-image-a}
    \caption{人間の知識創造のベクトル空間モデルにおける表現}
\end{figure}

\subsection{表とそのキャプションの例}
\begin{table}[htbp]
    \centering
    \begin{tabular}{|c|c|c|c|}
        \hline
        ML & n(特徴) & m(印象) & 性質 \\ \hline
        楽曲 (Hevner) & 6 & 8 & 特徴保存 \\ \hline
        カラーイメージスケール & 180 & 130 & 印象保存 \\ \hline
        音相理論 & 78 & 20 & 印象保存 \\ \hline
    \end{tabular}
    \caption{各MLの性質}
\end{table}

\section{数式の参照番号}
\begin{equation}
    S_1 = f(s_i) \quad 1 \leq i \leq n
\end{equation}
...

% Chapter 3: Hierarchy of Headings and Style
\chapter{見出しの階層構造とスタイルの設定}
\section{見出しのレベル}
\subsection{これはレベル3です}
...

% Chapter 4: Conclusion
\chapter{結論}
本論文では,...

% Acknowledgements
\chapter*{謝辞}
本研究の遂行にあたって,...

% References
\chapter*{参考文献}
\begin{thebibliography}{9}
    \bibitem{example1} 掛谷 英紀, 学問とは何か: 専門家・メディア・科学技術の倫理, 大学教育出版, 2005.
    \bibitem{example2} Y. Kiyoki, T. Kitagawa, T. Hayama, ``A metadatabase system for semantic image search by a mathematical model of meaning,'' ACM Sigmod Record, vol. 23, no. 4, pp. 34-41, 1994.
\end{thebibliography}

\end{document}
